\chapter{IP ranges for computer lab computers}% ===> this file was generated automatically by noweave --- better not edit it

So we have a quiz assignment in Canvas that we want the
students to do in a proctored setting (it's a KS). However, to prevent
them from doing it at home, or any other unproctored setting, during the time 
window of the exam, we have these options:

\begin{enumerate}
  \item Setting which IPs they can access the quiz from. Canvas allows for 
  this. Then I'll just list the IP ranges for the booked lab rooms. Hence, they 
  can't access the quiz from outside of the proctored environment.

  \item\label{CanvasExamRoom} Assign the quiz only to those who signed up. 
  However, it would still need the IP-blocking, since otherwise they can do it 
  from home anyway as long as they signed up.

  \item Just invalidate whoever submitted anything but wasn't recorded present 
  by the invigilators. This is a lot of manual work. And they can still leave 
  the room after 30 minutes and finish the quiz outside of the proctored 
  environment with aids that we don't allow.
\end{enumerate}

In our setting, the quiz was in the normal Canvas room of the course.
If we instead have a dedicated exam room in Canvas, that's just equivalent to 
option \ref{CanvasExamRoom} above.
To improve on option \ref{CanvasExamRoom} to not have to use an IP block, we 
need that the proctors note the time every student leaves so that we can 
compare the time they left to the time they submitted.


\section{Setting up the CLI}

We want to provide a command that takes arguments and options and passes those 
to a shell script.
Since we only provide one command, we must provide a function {\Tt{}add{\_}command\nwendquote} 
that we can use to add the command to any provided {\Tt{}Typer\nwendquote} instance {\Tt{}cli\nwendquote}.
\nwfilename{iprange.nw}\nwbegincode{1}\sublabel{NW2UbAm6-3pRHCY-1}\nwmargintag{{\nwtagstyle{}\subpageref{NW2UbAm6-3pRHCY-1}}}\moddef{iprange.py~{\nwtagstyle{}\subpageref{NW2UbAm6-3pRHCY-1}}}\endmoddef\nwstartdeflinemarkup\nwenddeflinemarkup
import typer
from typing import Annotated, List
\LA{}imports~{\nwtagstyle{}\subpageref{NW2UbAm6-1n4S18-1}}\RA{}

\LA{}argument and option definitions~{\nwtagstyle{}\subpageref{NW2UbAm6-2Ns6UP-1}}\RA{}

def add_command(cli):
  """
  Adds the [[iprange]] command to the given [[cli]].
  """
  \LA{}add the \code{}iprange\edoc{} command to \code{}cli\edoc{}~{\nwtagstyle{}\subpageref{NW2UbAm6-Uo1N5-1}}\RA{}

if __name__ == "__main__":
  cli = typer.Typer()
  add_command(cli)
  cli()
\nwnotused{iprange.py}\nwendcode{}\nwbegindocs{2}\nwdocspar


\section{The {\Tt{}iprange\nwendquote} command}

We want to provide a command that takes a list of lab rooms as argument and 
prints the IP ranges for them.
Now this is done the easiest with a shell script.
However, we want to be able to run it from this Python command-line utility 
too.
The main script is provided below as {\Tt{}\LA{}iprange.sh~{\nwtagstyle{}\subpageref{NW2UbAm6-ng2EQ-1}}\RA{}\nwendquote}.
We'll simply call it from our Python-implemented subommand.
\nwenddocs{}\nwbegincode{3}\sublabel{NW2UbAm6-Uo1N5-1}\nwmargintag{{\nwtagstyle{}\subpageref{NW2UbAm6-Uo1N5-1}}}\moddef{add the \code{}iprange\edoc{} command to \code{}cli\edoc{}~{\nwtagstyle{}\subpageref{NW2UbAm6-Uo1N5-1}}}\endmoddef\nwstartdeflinemarkup\nwusesondefline{\\{NW2UbAm6-3pRHCY-1}}\nwenddeflinemarkup
@cli.command()
def iprange(rooms: Annotated[List[str], rooms_arg]):
  """
  Generate the IP ranges for the given lab rooms.
  """
  \LA{}find the path of \code{}iprange.sh\edoc{}~{\nwtagstyle{}\subpageref{NW2UbAm6-2dxA1H-1}}\RA{}
  \LA{}run \code{}iprange.sh\edoc{} with the given \code{}rooms\edoc{}~{\nwtagstyle{}\subpageref{NW2UbAm6-4SFaGW-1}}\RA{}
\nwused{\\{NW2UbAm6-3pRHCY-1}}\nwendcode{}\nwbegincode{4}\sublabel{NW2UbAm6-2Ns6UP-1}\nwmargintag{{\nwtagstyle{}\subpageref{NW2UbAm6-2Ns6UP-1}}}\moddef{argument and option definitions~{\nwtagstyle{}\subpageref{NW2UbAm6-2Ns6UP-1}}}\endmoddef\nwstartdeflinemarkup\nwusesondefline{\\{NW2UbAm6-3pRHCY-1}}\nwenddeflinemarkup
rooms_arg = typer.Argument(help="The lab rooms to generate IP ranges for. "
                                "The lab room is the hostname prefix, eg red "
                                "(for Röd) or toke (for Toker).")
\nwused{\\{NW2UbAm6-3pRHCY-1}}\nwendcode{}\nwbegindocs{5}\nwdocspar

\subsection{Find the path of {\Tt{}iprange.sh\nwendquote}}

We'll use the {\Tt{}pkgutil\nwendquote} module to find the path of the shell script.
\nwenddocs{}\nwbegincode{6}\sublabel{NW2UbAm6-1n4S18-1}\nwmargintag{{\nwtagstyle{}\subpageref{NW2UbAm6-1n4S18-1}}}\moddef{imports~{\nwtagstyle{}\subpageref{NW2UbAm6-1n4S18-1}}}\endmoddef\nwstartdeflinemarkup\nwusesondefline{\\{NW2UbAm6-3pRHCY-1}}\nwprevnextdefs{\relax}{NW2UbAm6-1n4S18-2}\nwenddeflinemarkup
import pkgutil
\nwalsodefined{\\{NW2UbAm6-1n4S18-2}\\{NW2UbAm6-1n4S18-3}}\nwused{\\{NW2UbAm6-3pRHCY-1}}\nwendcode{}\nwbegindocs{7}\nwdocspar

We'll use the {\Tt{}get{\_}loader\nwendquote} function to get the path of the shell script.
{\Tt{}get{\_}loader\nwendquote} will return the path to the current module, that is, the 
{\Tt{}iprange\nwendquote} module and the {\Tt{}iprange.py\nwendquote} file.
So we'll get the parent to get the directory and then we just add the name of 
the shell script.
\nwenddocs{}\nwbegincode{8}\sublabel{NW2UbAm6-2dxA1H-1}\nwmargintag{{\nwtagstyle{}\subpageref{NW2UbAm6-2dxA1H-1}}}\moddef{find the path of \code{}iprange.sh\edoc{}~{\nwtagstyle{}\subpageref{NW2UbAm6-2dxA1H-1}}}\endmoddef\nwstartdeflinemarkup\nwusesondefline{\\{NW2UbAm6-Uo1N5-1}}\nwenddeflinemarkup
package_path = pathlib.Path(pkgutil.get_loader(__name__).path).parent
iprange_sh = package_path / "iprange.sh"
\nwused{\\{NW2UbAm6-Uo1N5-1}}\nwendcode{}\nwbegincode{9}\sublabel{NW2UbAm6-1n4S18-2}\nwmargintag{{\nwtagstyle{}\subpageref{NW2UbAm6-1n4S18-2}}}\moddef{imports~{\nwtagstyle{}\subpageref{NW2UbAm6-1n4S18-1}}}\plusendmoddef\nwstartdeflinemarkup\nwusesondefline{\\{NW2UbAm6-3pRHCY-1}}\nwprevnextdefs{NW2UbAm6-1n4S18-1}{NW2UbAm6-1n4S18-3}\nwenddeflinemarkup
import pathlib
\nwused{\\{NW2UbAm6-3pRHCY-1}}\nwendcode{}\nwbegindocs{10}\nwdocspar

\subsection{Run {\Tt{}iprange.sh\nwendquote} with the given {\Tt{}rooms\nwendquote}}

We'll use the {\Tt{}subprocess\nwendquote} module to run the shell script.
\nwenddocs{}\nwbegincode{11}\sublabel{NW2UbAm6-1n4S18-3}\nwmargintag{{\nwtagstyle{}\subpageref{NW2UbAm6-1n4S18-3}}}\moddef{imports~{\nwtagstyle{}\subpageref{NW2UbAm6-1n4S18-1}}}\plusendmoddef\nwstartdeflinemarkup\nwusesondefline{\\{NW2UbAm6-3pRHCY-1}}\nwprevnextdefs{NW2UbAm6-1n4S18-2}{\relax}\nwenddeflinemarkup
import subprocess
\nwused{\\{NW2UbAm6-3pRHCY-1}}\nwendcode{}\nwbegindocs{12}\nwdocspar

We'll use the {\Tt{}run\nwendquote} function to run the shell script.
We'll use the {\Tt{}check\nwendquote} argument to make it raise an exception if the shell 
script fails.
We don't need to capture the output, so we'll just redirect it to the standard 
output of the Python script.
\nwenddocs{}\nwbegincode{13}\sublabel{NW2UbAm6-4SFaGW-1}\nwmargintag{{\nwtagstyle{}\subpageref{NW2UbAm6-4SFaGW-1}}}\moddef{run \code{}iprange.sh\edoc{} with the given \code{}rooms\edoc{}~{\nwtagstyle{}\subpageref{NW2UbAm6-4SFaGW-1}}}\endmoddef\nwstartdeflinemarkup\nwusesondefline{\\{NW2UbAm6-Uo1N5-1}}\nwenddeflinemarkup
subprocess.run([iprange_sh, *rooms], check=True)
\nwused{\\{NW2UbAm6-Uo1N5-1}}\nwendcode{}\nwbegindocs{14}\nwdocspar


\section{The script {\Tt{}iprange.sh\nwendquote} to help with the address ranges}

This is a script
that generates a list of addresses for the given lab rooms.
We want to get the intervals, that is, the first and last address in a range 
that spans the lab room.
This way we can easily enter them in Canvas.
\nwenddocs{}\nwbegincode{15}\sublabel{NW2UbAm6-ng2EQ-1}\nwmargintag{{\nwtagstyle{}\subpageref{NW2UbAm6-ng2EQ-1}}}\moddef{iprange.sh~{\nwtagstyle{}\subpageref{NW2UbAm6-ng2EQ-1}}}\endmoddef\nwstartdeflinemarkup\nwenddeflinemarkup
#!/bin/bash
# Author: Daniel Bosk <dbosk@kth.se>
# License: MIT
# Description: Generates the address ranges for the given lab rooms.
# Usage: ./addresses.sh <lab room> <lab room> ...
# The lab room is the hostname prefix, eg red (for Röd) or toke (for Toker).

\LA{}constants~{\nwtagstyle{}\subpageref{NW2UbAm6-3UrjvT-1}}\RA{}
\LA{}helper functions~{\nwtagstyle{}\subpageref{NW2UbAm6-47YflN-1}}\RA{}

rooms=$*
\LA{}generate lab rooms CSV file~{\nwtagstyle{}\subpageref{NW2UbAm6-2Vvdxc-1}}\RA{}
for room in $rooms; do
  \LA{}print the address interval for \code{}room\edoc{}~{\nwtagstyle{}\subpageref{NW2UbAm6-vGl3a-1}}\RA{}
done
\nwnotused{iprange.sh}\nwendcode{}\nwbegindocs{16}\nwdocspar

\section{Generating a list of lab rooms' addresses}

We'll create a file that contains all addresses of the lab computers.
\nwenddocs{}\nwbegincode{17}\sublabel{NW2UbAm6-3UrjvT-1}\nwmargintag{{\nwtagstyle{}\subpageref{NW2UbAm6-3UrjvT-1}}}\moddef{constants~{\nwtagstyle{}\subpageref{NW2UbAm6-3UrjvT-1}}}\endmoddef\nwstartdeflinemarkup\nwusesondefline{\\{NW2UbAm6-ng2EQ-1}}\nwenddeflinemarkup
LABROOMS_CSV=$(mktemp)
\nwused{\\{NW2UbAm6-ng2EQ-1}}\nwendcode{}\nwbegincode{18}\sublabel{NW2UbAm6-2Vvdxc-1}\nwmargintag{{\nwtagstyle{}\subpageref{NW2UbAm6-2Vvdxc-1}}}\moddef{generate lab rooms CSV file~{\nwtagstyle{}\subpageref{NW2UbAm6-2Vvdxc-1}}}\endmoddef\nwstartdeflinemarkup\nwusesondefline{\\{NW2UbAm6-ng2EQ-1}}\nwenddeflinemarkup
list_lab_computer_hostnames_IPs $rooms > $LABROOMS_CSV
\nwused{\\{NW2UbAm6-ng2EQ-1}}\nwendcode{}\nwbegincode{19}\sublabel{NW2UbAm6-47YflN-1}\nwmargintag{{\nwtagstyle{}\subpageref{NW2UbAm6-47YflN-1}}}\moddef{helper functions~{\nwtagstyle{}\subpageref{NW2UbAm6-47YflN-1}}}\endmoddef\nwstartdeflinemarkup\nwusesondefline{\\{NW2UbAm6-ng2EQ-1}}\nwprevnextdefs{\relax}{NW2UbAm6-47YflN-2}\nwenddeflinemarkup
list_lab_computer_hostnames_IPs() \{
  local rooms=$*
  local room
  for room in $rooms; do
    \LA{}print the hostname-address pairs for \code{}room\edoc{}~{\nwtagstyle{}\subpageref{NW2UbAm6-1N0dK6-1}}\RA{}
  done
\}
\nwalsodefined{\\{NW2UbAm6-47YflN-2}}\nwused{\\{NW2UbAm6-ng2EQ-1}}\nwendcode{}\nwbegindocs{20}\nwdocspar

Now we just need to print the hostname-address pairs for each lab room.
We'll simply use the DNS to enumerate the computers.

There are two possible domains, {\Tt{}eecs.kth.se\nwendquote} (for the Unix computers) and 
{\Tt{}ug.kth.se\nwendquote} (for the Windows computers).
We'll try both and just ignore any results with {\Tt{}NXDOMAIN\nwendquote} in its output.

We'll also ignore any IPv6 addresses, that is, any results containing {\Tt{}IPv6\nwendquote} 
in its output.

We'll also assume that there will be less than 100 computer in a lab room.
All lab computers are named {\Tt{}<room>-<number>\nwendquote}, for instance 
{\Tt{}red-01.eecs.kth.se\nwendquote} or {\Tt{}toke-01.ug.kth.se\nwendquote}.
However, sometimes IT doesn't follow this pattern\footnote{%
  Thanks to Vahid for pointing this out!
}:
For instance, we have {\Tt{}mat01.ug.kth.se\nwendquote}.
\nwenddocs{}\nwbegincode{21}\sublabel{NW2UbAm6-1N0dK6-1}\nwmargintag{{\nwtagstyle{}\subpageref{NW2UbAm6-1N0dK6-1}}}\moddef{print the hostname-address pairs for \code{}room\edoc{}~{\nwtagstyle{}\subpageref{NW2UbAm6-1N0dK6-1}}}\endmoddef\nwstartdeflinemarkup\nwusesondefline{\\{NW2UbAm6-47YflN-1}}\nwenddeflinemarkup
for num in $(seq -w 1 99); do
  host $room-$num.eecs.kth.se | grep -v NXDOMAIN | grep -v IPv6 \\
    | cut -d " " -f 1,4
  host $room$num.eecs.kth.se | grep -v NXDOMAIN | grep -v IPv6 \\
    | cut -d " " -f 1,4
  host $room-$num.ug.kth.se | grep -v NXDOMAIN \\
    | cut -d " " -f 1,4
  host $room$num.ug.kth.se | grep -v NXDOMAIN \\
    | cut -d " " -f 1,4
done
\nwused{\\{NW2UbAm6-47YflN-1}}\nwendcode{}\nwbegindocs{22}\nwdocspar

\section{Getting the start and end address}

We'll use a file that contains all addresses of the lab room computers.
We find this file in {\Tt{}LABROOMS{\_}CSV\nwendquote}, as outlined above.
That file contains hostname and IP-address pairs.
We'll filter out (grep) the lines containing the lab room name, then we'll cut 
out the IP-addresses, and finally get the first ({\Tt{}head\nwendquote}) and last ({\Tt{}tail\nwendquote}).
We assume that they'll be reasonably in order.
In most cases they are, they might deviate if there is a problem, for instance 
that a computer has gotten a completely different IP.
\nwenddocs{}\nwbegincode{23}\sublabel{NW2UbAm6-47YflN-2}\nwmargintag{{\nwtagstyle{}\subpageref{NW2UbAm6-47YflN-2}}}\moddef{helper functions~{\nwtagstyle{}\subpageref{NW2UbAm6-47YflN-1}}}\plusendmoddef\nwstartdeflinemarkup\nwusesondefline{\\{NW2UbAm6-ng2EQ-1}}\nwprevnextdefs{NW2UbAm6-47YflN-1}{\relax}\nwenddeflinemarkup
get_start_end_address() \{
  local room=$1
  local addresses=$(grep -i $room $LABROOMS_CSV | cut -d " " -f 2)
  local start=$(echo "$addresses" | head -n 1)
  local end=$(echo "$addresses" | tail -n 1)
  echo $start $end
\}
\nwused{\\{NW2UbAm6-ng2EQ-1}}\nwendcode{}\nwbegindocs{24}\nwdocspar

This leaves us with the following.
\nwenddocs{}

\nwixlogsorted{c}{{add the \code{}iprange\edoc{} command to \code{}cli\edoc{}}{NW2UbAm6-Uo1N5-1}{\nwixu{NW2UbAm6-3pRHCY-1}\nwixd{NW2UbAm6-Uo1N5-1}}}%
\nwixlogsorted{c}{{argument and option definitions}{NW2UbAm6-2Ns6UP-1}{\nwixu{NW2UbAm6-3pRHCY-1}\nwixd{NW2UbAm6-2Ns6UP-1}}}%
\nwixlogsorted{c}{{constants}{NW2UbAm6-3UrjvT-1}{\nwixu{NW2UbAm6-ng2EQ-1}\nwixd{NW2UbAm6-3UrjvT-1}}}%
\nwixlogsorted{c}{{find the path of \code{}iprange.sh\edoc{}}{NW2UbAm6-2dxA1H-1}{\nwixu{NW2UbAm6-Uo1N5-1}\nwixd{NW2UbAm6-2dxA1H-1}}}%
\nwixlogsorted{c}{{generate lab rooms CSV file}{NW2UbAm6-2Vvdxc-1}{\nwixu{NW2UbAm6-ng2EQ-1}\nwixd{NW2UbAm6-2Vvdxc-1}}}%
\nwixlogsorted{c}{{helper functions}{NW2UbAm6-47YflN-1}{\nwixu{NW2UbAm6-ng2EQ-1}\nwixd{NW2UbAm6-47YflN-1}\nwixd{NW2UbAm6-47YflN-2}}}%
\nwixlogsorted{c}{{imports}{NW2UbAm6-1n4S18-1}{\nwixu{NW2UbAm6-3pRHCY-1}\nwixd{NW2UbAm6-1n4S18-1}\nwixd{NW2UbAm6-1n4S18-2}\nwixd{NW2UbAm6-1n4S18-3}}}%
\nwixlogsorted{c}{{iprange.py}{NW2UbAm6-3pRHCY-1}{\nwixd{NW2UbAm6-3pRHCY-1}}}%
\nwixlogsorted{c}{{iprange.sh}{NW2UbAm6-ng2EQ-1}{\nwixd{NW2UbAm6-ng2EQ-1}}}%
\nwixlogsorted{c}{{print the address interval for \code{}room\edoc{}}{NW2UbAm6-vGl3a-1}{\nwixu{NW2UbAm6-ng2EQ-1}\nwixd{NW2UbAm6-vGl3a-1}}}%
\nwixlogsorted{c}{{print the hostname-address pairs for \code{}room\edoc{}}{NW2UbAm6-1N0dK6-1}{\nwixu{NW2UbAm6-47YflN-1}\nwixd{NW2UbAm6-1N0dK6-1}}}%
\nwixlogsorted{c}{{run \code{}iprange.sh\edoc{} with the given \code{}rooms\edoc{}}{NW2UbAm6-4SFaGW-1}{\nwixu{NW2UbAm6-Uo1N5-1}\nwixd{NW2UbAm6-4SFaGW-1}}}%
\nwbegincode{25}\sublabel{NW2UbAm6-vGl3a-1}\nwmargintag{{\nwtagstyle{}\subpageref{NW2UbAm6-vGl3a-1}}}\moddef{print the address interval for \code{}room\edoc{}~{\nwtagstyle{}\subpageref{NW2UbAm6-vGl3a-1}}}\endmoddef\nwstartdeflinemarkup\nwusesondefline{\\{NW2UbAm6-ng2EQ-1}}\nwenddeflinemarkup
get_start_end_address $room
\nwused{\\{NW2UbAm6-ng2EQ-1}}\nwendcode{}
