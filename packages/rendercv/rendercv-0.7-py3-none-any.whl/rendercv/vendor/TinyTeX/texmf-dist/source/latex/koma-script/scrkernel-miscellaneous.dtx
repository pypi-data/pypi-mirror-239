% \iffalse meta-comment
% ======================================================================
% scrkernel-miscellaneous.dtx
% Copyright (c) Markus Kohm, 2002-2023
%
% This file is part of the LaTeX2e KOMA-Script bundle.
%
% This work may be distributed and/or modified under the conditions of
% the LaTeX Project Public License, version 1.3c of the license.
% The latest version of this license is in
%   http://www.latex-project.org/lppl.txt
% and version 1.3c or later is part of all distributions of LaTeX 
% version 2005/12/01 or later and of this work.
%
% This work has the LPPL maintenance status "author-maintained".
%
% The Current Maintainer and author of this work is Markus Kohm.
%
% This work consists of all files listed in MANIFEST.md.
% ======================================================================
%%% From File: $Id: scrkernel-miscellaneous.dtx 4032 2023-04-17 09:45:11Z kohm $
%<init>%%%            (run: init)
%<option>%%%            (run: option)
%<execoption>%%%            (run: execoption)
%<body>%%%            (run: body)
%<*dtx>
\ifx\ProvidesFile\undefined\def\ProvidesFile#1[#2]{}\fi
\begingroup
  \def\filedate$#1: #2-#3-#4 #5${\gdef\filedate{#2/#3/#4}}
  \filedate$Date: 2023-04-17 11:45:11 +0200 (Mo, 17. Apr 2023) $
  \def\filerevision$#1: #2 ${\gdef\filerevision{r#2}}
  \filerevision$Revision: 4032 $
  \edef\reserved@a{%
    \noexpand\endgroup
    \noexpand\ProvidesFile{scrkernel-miscellaneous.dtx}%
                          [\filedate\space\filerevision\space
                           KOMA-Script source
                           (miscellaneous, maybe everything or nothing)]
  }%
\reserved@a
\documentclass[USenglish]{koma-script-source-doc}
\usepackage{babel}
\setcounter{StandardModuleDepth}{2}
\begin{document}
\DocInput{scrkernel-miscellaneous.dtx}
\end{document}
%</dtx>
% \fi
%
% \changes{v2.95}{2002/07/01}{first version after splitting \file{scrclass.dtx}}
% \changes{v3.36}{2022/03/01}{switch over from \cls*{scrdoc} to
%   \cls*{koma-script-source-doc}}
% \changes{v3.36}{2022/03/01}{whole implementation documentation in English}
% \changes{v3.40}{2023/04/17}{guide names changed}
%
% \GetFileInfo{scrkernel-miscellaneous.dtx}
% \title{All the Code of
%   \href{https://komascript.de}{\KOMAScript} not Fitting into another File}
% \author{\href{mailto:komascript@gmx.info}{Markus Kohm}}
% \date{Revision \fileversion{} of \filedate}
% \maketitle
% \begin{abstract}
%   Currently \file{scrkernel-miscellaneous.dtx} is a mess of miscellaneous
%   code. There is a big, fat \textbf{ToDo} to change this!
% \end{abstract}
% \tableofcontents
%
% \section{User Manual}
%
% You can find the user documentation the commands implemented here in the
% \KOMAScript{} manual, either the German \file{scrguide-de.pdf} or the
% English \file{scrguide-en.pdf}.
%
% \MaybeStop{\PrintIndex}
%
%
% \section{Implementation of Special Requirements of Package \pkg*{scrletter}}
%
% The \pkg*{scrletter} depends on some features of the \KOMAScript{}
% classes. However, a special user request was to make it work with the
% standard classes too. So we need additional code to detect not using a
% \KOMAScript{} class and to make this work.
% \begin{description}
% \item[ToDo:] Can we move this code into a letter file or use a new file only
%   used for package \pkg*{scrletter}?
% \end{description}
%    \begin{macrocode}
%<*package&letter&init>
\scr@ifundefinedorrelax{KOMAClassName}{%
  \PackageWarningNoLine{scrletter}{%
    Cannot detect any KOMA-Script class.\MessageBreak
    Package `scrletter' is an extension to\MessageBreak
    KOMA-Script classes.\MessageBreak
    Usage with another class is not recommended%
    \@ifclassloaded{article}{}{%
      \@ifclassloaded{report}{}{%
        \@ifclassloaded{book}{}{%
          \MessageBreak and can result in several errors%
        }%
      }%
    }%
  }%
  \RequirePackage{scrextend}%
%    \end{macrocode}
%
% \begin{command}{\setparsizes}
% \changes{v3.23}{2017/02/22}{simple fallback definition}
% \begin{macro}{\par@updaterelative}
% \changes{v3.23}{2017/02/22}{empty fallback definition}
% If \pkg*{scrletter} is not used with a \KOMAScript{} class \cs{setparsizes}
% would not be available. But it is used in \cs{vb@t@z}, \cs{@addrfield},
% \cs{@datefield}, \cs{@makelettertitle}, \cs{@subjfield}, \cs{opening},
% \cs{closing}. Instead of changing the implementation of all these commands
% and macros, we define a very simple version of \cs{setparsizes}. And in case
% of using this simple version \cs{par@updaterelative} would become a
% do-nothing macro. This does work in case of the commands and macros listed
% above. But maybe it does not work in other cases.
% \begin{description}
% \item[ToDo:] Can we move this code into \file{scrkernel-paragraphs.dtx} or
%   use a new file only for package \pkg*{scrletter}?
% \end{description}
%    \begin{macrocode}
  \providecommand*{\setparsizes}[3]{%
    \setlength{\parindent}{#1}%
    \setlength{\parskip}{#2}%
    \setlength{\parfillskip}{#3}%
  }
  \providecommand*{\par@updaterelative}{}
%    \end{macrocode}
% \end{macro}^^A \par@updaterelative
% \end{command}^^A \setparsizes
%    \begin{macrocode}
}{}
%</package&letter&init>
%    \end{macrocode}
%
% \begin{macro}{\scr@renewforletter,\@renewforletter,\do@renewforletter@list,
%               \do@renewforletter}
% \changes{v3.20}{2014/04/10}{added}
% \changes{v3.25}{2017/11/15}{removed}
% \end{macro}
%
%
% \section{Implementation of the Draft Mode}
%
% The \KOMAScript{} classes provide two draft options, the general one and a
% more specific.
% \begin{description}
% \item[ToDo:] Should we move this to \file{scrkernel-paragraphs.dtx}?
% \end{description}
% 
% \begin{option}{draft}
% \changes{v2.95}{2002/12/05}{re-implemented using \pkg{keyval}}
% \changes{v3.17}{2015/03/12}{using internal value storage}
% \changes{v3.25}{2017/11/21}{respect option \opt{overfullrule}}
% \changes{v3.12}{2013/03/05}{using \cs{FamilyKeyState}}
% \changes{v3.39}{2022/11/11}{initial dot removed from member argument of
%   option storage commands}
% \begin{option}{final}
% \changes{v2.95}{2002/12/05}{obsolete}
% \changes{v2.97d}{2007/10/03}{\cs{PackageInfo} replaced by
%   \cs{PackageInfoNoLine}}
% \changes{v3.01a}{2008/11/21}{standard instead of obsolete}
% \changes{v3.02c}{2008/02/17}{explicit optional argument for \pkg*{scrextend}}
% In draft mode classes usually show a rule for overfull boxes. Despite this
% is usually already handled by the class, package \pkg*{scrextend} also
% provides the option, because of correct setting of the internal value
% storage also of option \opt{overfullrule}.
%    \begin{macrocode}
%<*(class|extend)&option>
\KOMA@key{draft}[true]{%
  \KOMA@set@ifkey{draft}{@tempswa}{#1}%
  \ifx\FamilyKeyState\FamilyKeyStateProcessed
    \if@tempswa
      \KOMA@kav@replacebool{%
%<class>        \KOMAClassFileName
%<package&extend>        scrextend.\scr@pkgextension
      }{draft}{@tempswa}%
      \KOMA@kav@replacebool{%
%<class>        \KOMAClassFileName
%<package&extend>        scrextend.\scr@pkgextension
      }{overfullrule}{@tempswa}%
      \setlength{\overfullrule}{5pt}%
    \else
      \KOMA@kav@replacebool{%
%<class>        \KOMAClassFileName
%<package&extend>        scrextend.\scr@pkgextension
      }{draft}{@tempswa}%
      \KOMA@kav@replacebool{%
%<class>        \KOMAClassFileName
%<package&extend>        scrextend.\scr@pkgextension
      }{overfullrule}{@tempswa}%
      \setlength{\overfullrule}{\z@}%
    \fi
  \fi
}
%<class>\KOMA@kav@add{\KOMAClassFileName}{draft}{false}
%<*package&extend>
\KOMA@kav@xadd{scrextend.\scr@pkgextension}{draft}{%
  \ifdim\overfullrule>\z@ true\else false\fi
}
%</package&extend>
\KOMA@DeclareStandardOption
%<package&extend>[scrextend]
{final}{draft=false}
%    \end{macrocode}
% \begin{option}{overfullrule}
% \changes{v3.25}{2017/11/21}{added}
% \changes{v3.39}{2022/11/11}{initial dot removed from member argument of
%   option storage commands}
% Because option \opt{draft} can result in unwanted effects I've added a
% new option to explicite only change the overfull rule setting. However for
% \textsf{scrextend} and the \KOMAScript{} classes the result is the same.
%    \begin{macrocode}
\KOMA@key{overfullrule}[true]{%
  \KOMA@set@ifkey{overfullrule}{@tempswa}{#1}%
  \ifx\FamilyKeyState\FamilyKeyStateProcessed
    \if@tempswa
      \KOMA@kav@replacebool{%
%<class>        \KOMAClassFileName
%<package&extend>        scrextend.\scr@pkgextension
      }{draft}{@tempswa}%
      \KOMA@kav@replacebool{%
%<class>        \KOMAClassFileName
%<package&extend>        scrextend.\scr@pkgextension
      }{overfullrule}{@tempswa}%
      \setlength{\overfullrule}{5pt}%
    \else
      \KOMA@kav@replacebool{%
%<class>        \KOMAClassFileName
%<package&extend>        scrextend.\scr@pkgextension
      }{draft}{@tempswa}%
      \KOMA@kav@replacebool{%
%<class>        \KOMAClassFileName
%<package&extend>        scrextend.\scr@pkgextension
      }{overfullrule}{@tempswa}%
      \setlength{\overfullrule}{\z@}%
    \fi
  \fi
}
%<class>\KOMA@kav@add{\KOMAClassFileName}{overfullrule}{false}
%<*package&extend>
\KOMA@kav@xadd{scrextend.\scr@pkgextension}{overfullrule}{%
  \ifdim\overfullrule>\z@ true\else false\fi
}
%</package&extend>
%</(class|extend)&option>
%    \end{macrocode}
% \end{option}^^A overfullrule
% \end{option}^^A final
% \end{option}^^A draft
%
%
% \section{Implementations for the Math Mode}
%
% \KOMAScript{} does not make something special for the math mode.
% \begin{description}
% \item[ToDo:] Should we move this to the special treatment of \pkg{amsmath}
%   in \file{scrkernel-basics.dtx} or move it with the special treatment of
%   \pkg{amsmath} in \file{scrkernel-basics.dtx} to a new file
%   \file{scrkernel-math.dtx}?
% \end{description}
%
% \begin{option}{leqno,fleqno}
% The \KOMAScript{} classes but \cls*{scrlttr2} do support the standard
% class options \opt{leqno} and \opt{fleqn}.
%    \begin{macrocode}
%<*class&!letter&option>
\DeclareOption{leqno}{%%
%% This is file `leqno.sty',
%% generated with the docstrip utility.
%%
%% The original source files were:
%%
%% latex209.dtx  (with options: `leqno')
%% 
%% This is a generated file.
%% 
%% The source is maintained by the LaTeX Project team and bug
%% reports for it can be opened at https://latex-project.org/bugs.html
%% (but please observe conditions on bug reports sent to that address!)
%% 
%% 
%% Copyright (C) 1993-2023
%% The LaTeX Project and any individual authors listed elsewhere
%% in this file.
%% 
%% This file was generated from file(s) of the LaTeX base system.
%% --------------------------------------------------------------
%% 
%% It may be distributed and/or modified under the
%% conditions of the LaTeX Project Public License, either version 1.3c
%% of this license or (at your option) any later version.
%% The latest version of this license is in
%%    https://www.latex-project.org/lppl.txt
%% and version 1.3c or later is part of all distributions of LaTeX
%% version 2008 or later.
%% 
%% This file has the LPPL maintenance status "maintained".
%% 
%% This file may only be distributed together with a copy of the LaTeX
%% base system. You may however distribute the LaTeX base system without
%% such generated files.
%% 
%% The list of all files belonging to the LaTeX base distribution is
%% given in the file `manifest.txt'. See also `legal.txt' for additional
%% information.
%% 
%% The list of derived (unpacked) files belonging to the distribution
%% and covered by LPPL is defined by the unpacking scripts (with
%% extension .ins) which are part of the distribution.
\@obsoletefile{leqno.clo}{leqno.sty}
%%
%% This is file `leqno.sty',
%% generated with the docstrip utility.
%%
%% The original source files were:
%%
%% latex209.dtx  (with options: `leqno')
%% 
%% This is a generated file.
%% 
%% The source is maintained by the LaTeX Project team and bug
%% reports for it can be opened at https://latex-project.org/bugs.html
%% (but please observe conditions on bug reports sent to that address!)
%% 
%% 
%% Copyright (C) 1993-2023
%% The LaTeX Project and any individual authors listed elsewhere
%% in this file.
%% 
%% This file was generated from file(s) of the LaTeX base system.
%% --------------------------------------------------------------
%% 
%% It may be distributed and/or modified under the
%% conditions of the LaTeX Project Public License, either version 1.3c
%% of this license or (at your option) any later version.
%% The latest version of this license is in
%%    https://www.latex-project.org/lppl.txt
%% and version 1.3c or later is part of all distributions of LaTeX
%% version 2008 or later.
%% 
%% This file has the LPPL maintenance status "maintained".
%% 
%% This file may only be distributed together with a copy of the LaTeX
%% base system. You may however distribute the LaTeX base system without
%% such generated files.
%% 
%% The list of all files belonging to the LaTeX base distribution is
%% given in the file `manifest.txt'. See also `legal.txt' for additional
%% information.
%% 
%% The list of derived (unpacked) files belonging to the distribution
%% and covered by LPPL is defined by the unpacking scripts (with
%% extension .ins) which are part of the distribution.
\@obsoletefile{leqno.clo}{leqno.sty}
%%
%% This is file `leqno.sty',
%% generated with the docstrip utility.
%%
%% The original source files were:
%%
%% latex209.dtx  (with options: `leqno')
%% 
%% This is a generated file.
%% 
%% The source is maintained by the LaTeX Project team and bug
%% reports for it can be opened at https://latex-project.org/bugs.html
%% (but please observe conditions on bug reports sent to that address!)
%% 
%% 
%% Copyright (C) 1993-2023
%% The LaTeX Project and any individual authors listed elsewhere
%% in this file.
%% 
%% This file was generated from file(s) of the LaTeX base system.
%% --------------------------------------------------------------
%% 
%% It may be distributed and/or modified under the
%% conditions of the LaTeX Project Public License, either version 1.3c
%% of this license or (at your option) any later version.
%% The latest version of this license is in
%%    https://www.latex-project.org/lppl.txt
%% and version 1.3c or later is part of all distributions of LaTeX
%% version 2008 or later.
%% 
%% This file has the LPPL maintenance status "maintained".
%% 
%% This file may only be distributed together with a copy of the LaTeX
%% base system. You may however distribute the LaTeX base system without
%% such generated files.
%% 
%% The list of all files belonging to the LaTeX base distribution is
%% given in the file `manifest.txt'. See also `legal.txt' for additional
%% information.
%% 
%% The list of derived (unpacked) files belonging to the distribution
%% and covered by LPPL is defined by the unpacking scripts (with
%% extension .ins) which are part of the distribution.
\@obsoletefile{leqno.clo}{leqno.sty}
\input{leqno.clo}
\endinput
%%
%% End of file `leqno.sty'.

\endinput
%%
%% End of file `leqno.sty'.

\endinput
%%
%% End of file `leqno.sty'.
}
\DeclareOption{fleqn}{%%
%% This is file `fleqn.sty',
%% generated with the docstrip utility.
%%
%% The original source files were:
%%
%% latex209.dtx  (with options: `fleqn')
%% 
%% This is a generated file.
%% 
%% The source is maintained by the LaTeX Project team and bug
%% reports for it can be opened at https://latex-project.org/bugs.html
%% (but please observe conditions on bug reports sent to that address!)
%% 
%% 
%% Copyright (C) 1993-2023
%% The LaTeX Project and any individual authors listed elsewhere
%% in this file.
%% 
%% This file was generated from file(s) of the LaTeX base system.
%% --------------------------------------------------------------
%% 
%% It may be distributed and/or modified under the
%% conditions of the LaTeX Project Public License, either version 1.3c
%% of this license or (at your option) any later version.
%% The latest version of this license is in
%%    https://www.latex-project.org/lppl.txt
%% and version 1.3c or later is part of all distributions of LaTeX
%% version 2008 or later.
%% 
%% This file has the LPPL maintenance status "maintained".
%% 
%% This file may only be distributed together with a copy of the LaTeX
%% base system. You may however distribute the LaTeX base system without
%% such generated files.
%% 
%% The list of all files belonging to the LaTeX base distribution is
%% given in the file `manifest.txt'. See also `legal.txt' for additional
%% information.
%% 
%% The list of derived (unpacked) files belonging to the distribution
%% and covered by LPPL is defined by the unpacking scripts (with
%% extension .ins) which are part of the distribution.
\@obsoletefile{fleqn.clo}{fleqn.sty}
%%
%% This is file `fleqn.sty',
%% generated with the docstrip utility.
%%
%% The original source files were:
%%
%% latex209.dtx  (with options: `fleqn')
%% 
%% This is a generated file.
%% 
%% The source is maintained by the LaTeX Project team and bug
%% reports for it can be opened at https://latex-project.org/bugs.html
%% (but please observe conditions on bug reports sent to that address!)
%% 
%% 
%% Copyright (C) 1993-2023
%% The LaTeX Project and any individual authors listed elsewhere
%% in this file.
%% 
%% This file was generated from file(s) of the LaTeX base system.
%% --------------------------------------------------------------
%% 
%% It may be distributed and/or modified under the
%% conditions of the LaTeX Project Public License, either version 1.3c
%% of this license or (at your option) any later version.
%% The latest version of this license is in
%%    https://www.latex-project.org/lppl.txt
%% and version 1.3c or later is part of all distributions of LaTeX
%% version 2008 or later.
%% 
%% This file has the LPPL maintenance status "maintained".
%% 
%% This file may only be distributed together with a copy of the LaTeX
%% base system. You may however distribute the LaTeX base system without
%% such generated files.
%% 
%% The list of all files belonging to the LaTeX base distribution is
%% given in the file `manifest.txt'. See also `legal.txt' for additional
%% information.
%% 
%% The list of derived (unpacked) files belonging to the distribution
%% and covered by LPPL is defined by the unpacking scripts (with
%% extension .ins) which are part of the distribution.
\@obsoletefile{fleqn.clo}{fleqn.sty}
%%
%% This is file `fleqn.sty',
%% generated with the docstrip utility.
%%
%% The original source files were:
%%
%% latex209.dtx  (with options: `fleqn')
%% 
%% This is a generated file.
%% 
%% The source is maintained by the LaTeX Project team and bug
%% reports for it can be opened at https://latex-project.org/bugs.html
%% (but please observe conditions on bug reports sent to that address!)
%% 
%% 
%% Copyright (C) 1993-2023
%% The LaTeX Project and any individual authors listed elsewhere
%% in this file.
%% 
%% This file was generated from file(s) of the LaTeX base system.
%% --------------------------------------------------------------
%% 
%% It may be distributed and/or modified under the
%% conditions of the LaTeX Project Public License, either version 1.3c
%% of this license or (at your option) any later version.
%% The latest version of this license is in
%%    https://www.latex-project.org/lppl.txt
%% and version 1.3c or later is part of all distributions of LaTeX
%% version 2008 or later.
%% 
%% This file has the LPPL maintenance status "maintained".
%% 
%% This file may only be distributed together with a copy of the LaTeX
%% base system. You may however distribute the LaTeX base system without
%% such generated files.
%% 
%% The list of all files belonging to the LaTeX base distribution is
%% given in the file `manifest.txt'. See also `legal.txt' for additional
%% information.
%% 
%% The list of derived (unpacked) files belonging to the distribution
%% and covered by LPPL is defined by the unpacking scripts (with
%% extension .ins) which are part of the distribution.
\@obsoletefile{fleqn.clo}{fleqn.sty}
\input{fleqn.clo}
\endinput
%%
%% End of file `fleqn.sty'.

\endinput
%%
%% End of file `fleqn.sty'.

\endinput
%%
%% End of file `fleqn.sty'.
}
%</class&!letter&option>
%    \end{macrocode}
% \end{option}
%
% \begin{counter}{equation}
% \changes{v3.37}{2022/08/24}{accidentally removed \cs{@addtoreset} added
%   again}
% \begin{command}{\theequation}
% \changes{v2.97e}{2007/11/23}{chapter numbers only in main matter}
% \changes{v3.27}{2022/03/01}{\cs{relax} replaced by percent because of
%   \pkg*{hyperref}}
% The \KOMAScript{} classes \cls*{scrbook} and \cls*{scrreprt} have a special
% treatment for equation numbers. For \cls*{scrreprt} the number just depends
% on the chapter number. But for \cls*{scrbook} it also depends on the active
% matter.
% \begin{description}
% \item[ToDo:] With \cs{counterwidth} instead of \cs{@addtoreset} the
%   \cls*{scrreprt} code could be very simple. But currently we still support
%   old \LaTeX{} kernels without \cs{counterwithin}.
% \item[ToDo:] Should we also add a definition for \cls*{scrartcl} as done for
%   \cls*{article}, though the same definition is already the default of the
%   \LaTeX{} kernel?
% \end{description}
%    \begin{macrocode}
%<*(book|report)&body>
\@addtoreset{equation}{chapter}
\renewcommand*{\theequation}{%
%<*book>
%<!v4>  \expandafter\ifnum\scr@v@is@gt{2.97d}%
    \if@mainmatter\thechapter.\fi
%<!v4>  \else
%</book>
%<!v4>    \thechapter.%
%<book&!v4>  \fi
  \@arabic\c@equation
}

%</(book|report)&body>
%    \end{macrocode}
% \end{command}^^A \theequation
% \end{counter}^^A equation
%
%
% \section{Implementation of the Execution and Processing of Options}
%
% After the definition of all options, some defaults have to be executed and
% all options have to be processed.
% \begin{description}
% \item[ToDo:] Can we move the execution of default to the files, where the
%   corresponding options are defined?
% \item[ToDo:] Can we move the processing of options to
%   \file{scrkernel-basics.dtx} where the command itself is defined?
% \end{description}
%
% \changes{v2.97c}{2007/04/19}{using \cs{KOMAExecuteOptions} instead of
%   \cs{KOMAoptions} and \cs{KOMAProcessOptions} instead of
%   \cs{ProcessOptions}}
% \changes{v2.97c}{2007/04/19}{options that are implicit defaults are not set
%   explicitly}
% \changes{v2.97c}{2007/07/18}{removed \opt{twoside} from the default}
% \changes{v2.98c}{2008/03/10}{\opt{headings\quotechar=big} added to the
%   default}
% \changes{v3.11c}{2013/03/04}{\opt{headings\quotechar=big} not with
%   \cls*{scrlttr2}}
%    \begin{macrocode}
%<*execoption>
%<*class>
\KOMAExecuteOptions{%
  parskip=false,%
%<!letter>  headings=big%
}
%</class>
\KOMAProcessOptions\relax
%</execoption>
%    \end{macrocode}
%
%
% \section{Implementation of Temporary Switch to Single Column Mode}
%
% Sometimes we need to temporary switch to single column mode, e.g., to print
% a table of contents or similar. This is not needed for \cls*{scrlttr2} or
% \pkg*{scrletter}.
% \begin{description}
% \item[ToDo:] Should we move this to \file{scrkernel-paragraphs.dtx} or
%   another file? Currently it is used in \file{scrkernel-index.dtx},
%   \file{scrkernel-notepaper.dtx}, and \file{scrkernel-title.dtx} (and also
%   in \file{scrhack.dtx}, which also defines it of needed).
% \end{description}
% 
% \begin{macro}{\@restonecolfalse,\@restonecoltrue,\if@restonecol}
% We need a local switch to be able to restore two column mode after locally
% switching to one column mode (or visa versa). For \pkg*{scrextend} this is a
% little bit complicated, because the switch could already exist in the used
% class.
%    \begin{macrocode}
%<*!letter&body>
%<*!extend>
\newif\if@restonecol
%</!extend>
%<*extend>
\scr@ifundefinedorrelax{if@restonecol}{%
  \expandafter\newif\csname if@restonecol\endcsname
  \@restonecolfalse
}{}
%</extend>
%</!letter&body>
%    \end{macrocode}
% \end{macro}
% 
% \section{Implementation of the Dictum}
%
% Printing a \emph{dictum} is also not supported by letters.
% \begin{description}
% \item[ToDo:] Should we move this to \file{scrkernel-sections.dtx} or to a
%   new file?
% \end{description}
% 
% \begin{command}{\dictum}
% \changes{v2.8q}{2001/20/04}{added}
% \changes{v2.9l}{2003/01/28}{\cs{edef} replaced by \cs{def} ersetzt}
% \changes{v2.9q}{2004/04/19}{group added}
% \begin{command}{\dictumwidth,
%                 \raggeddictum,\raggeddictumtext,\raggeddictumauthor}
% \changes{v2.8q}{2001/20/04}{added}
% \begin{fontelement}{dictum}
% \changes{v2.8q}{2001/20/04}{added}
% \changes{v3.20}{2016/01/29}{\cs{sffamily} replaced by \cs{@gr@gsffamily}}
% \changes{v3.39}{2023/04/04}{\cs{@gr@gsfamily} replaced by \cs{maybesffamily}}
% \begin{fontelement}{dictumtext,dictumauthor}
% \changes{v2.8q}{2001/20/04}{added}
% \begin{command}{\dictumauthorformat}
% \changes{v2.8q}{2001/20/04}{added}
% \begin{command}{\dictumrule}
% \changes{v3.10}{2011/09/02}{added}
% The \emph{dictum} consist of a text, the dictum, and an optional
% author. Text and author are separated by a horizontal rule.
%    \begin{macrocode}
%<*!letter&body>
\newcommand{\dictum}[2][]{\par%
  \begingroup
    \raggeddictum\parbox{\dictumwidth}{%
      {\@dictumfont{\raggeddictumtext #2\strut\par}%
        \def\@tempa{#1}\ifx\@tempa\@empty\else%
          {\raggeddictumtext\dictumrule}%
          \raggeddictumauthor\@dictumauthorfont\dictumauthorformat{#1}%
          \strut\par%
        \fi%
      }%
    }\par%
  \endgroup
}
\newcommand*{\dictumrule}{\vskip-1ex\hrulefill\par}%
\newcommand*{\dictumwidth}{.3333\textwidth}
\newcommand*{\raggeddictum}{\raggedleft}
\newcommand*{\dictumauthorformat}[1]{(#1)}
\newcommand*{\raggeddictumtext}{\raggedright}
\newcommand*{\raggeddictumauthor}{\raggedleft}
\newkomafont{dictum}{\normalfont\normalcolor\maybesffamily\small}
\newkomafont{dictumauthor}{\itshape}
\aliaskomafont{dictumtext}{dictum}
%</!letter&body>
%    \end{macrocode}
% \end{command}^^A \dictumrule
% \end{command}^^A \dictumauthorformat
% \end{fontelement}^^A dictumtext,dictumauthor
% \end{fontelement}^^A dictum
% \end{command}^^A \dictumwidth,\raggeddictum…\raggeddictumauthor
% \end{command}^^A \dictum
%
%
% \section{Setting Default Lengths of \env{minipage} and Frames}
%
% \begin{description}
% \item[ToDo:] Should we move the default for \len{@mpfootins} to
%   \file{scrkernel-footnotes.dtx}?
% \item[ToDo:] Do we really need to set initial values to \len{fboxsep} and
% \len{fboxrule}? The same defaults using |pt| instead of \cs{p@} are already
% used in the \LaTeX{} kernel. Wouldn't that be enough?
% \end{description}
%
% \begin{length}{@mpfootins}
% \begin{length}{fboxsep,fboxrule}
% We are using the same settings like the standard classes.
%    \begin{macrocode}
%<*class&body>
\setlength{\skip\@mpfootins}{\skip\footins}
\setlength\fboxsep{3\p@}
\setlength\fboxrule{.4\p@}
%</class&body>
%    \end{macrocode}
% \end{length}
% \end{length}
%
%
% \section{Implementation of Detection of Odd and Even Pages}
%
% \begin{description}
% \item[ToDo:] Should this code moved to \file{scrkernel-basics.dtx} or to a
%   new one?
% \end{description}
%
% \begin{command}{\ifthispageodd,\Ifthispageodd}
% \changes{v2.8q}{2001/11/22}{added}
% \changes{v3.28}{2019/11/19}{\cs{ifthispageodd} renamed to \cs{Ifthispageodd}}
% Users often use |\ifodd\value{page}| to test, whether the current page is
% odd or even. But this is dramatically wrong, because of the asynchronous
% output of \TeX. So here we define a command
% \begin{quote}
%   \Ifthispageodd\marg{then code}\marg{else code}
% \end{quote}
% that internally uses a label-like mechanism to do the decision. We do not
% really use \cs{label} and \cs{pageref}, because this would not work with
% Roman or alphabetic page numbers. So we use a direct write to the \file{aux}
% file instead.
%
% \begin{macro}{\new@tpo@label}
% \changes{v2.8q}{2001/11/22}{added}
% \changes{v2.95b}{2006/07/28}{using \cs{@newl@bel}}
% \changes{v3.20}{2016/03/25}{use after definition in the \file{aux}-file}
% Definition of a new \textbf{t}rue-\textbf{p}age-\textbf{o}dd label. To be
% used in the \file{aux} file. A fallback definition is also added to the
% \file{aux} file, so there wouldn't be an error, e.g., after removing package
% \pkg*{scrextend} from a document.
%    \begin{macrocode}
%<*(class|extend)&body>
\newcommand*{\new@tpo@label}{\@newl@bel{tpo}}
\AtBeginDocument{%
  \if@filesw
    \immediate\write\@auxout{%
      \string\providecommand*\string\new@tpo@label[2]{}%
    }%
  \fi
}
%    \end{macrocode}
% \end{macro}^^A \new@tpo@label
%
% \begin{macro}{\scr@tpo}
% \changes{v2.8q}{2001/11/22}{added}
% We need a kind of counter with the actual usage number of
% \cs{Ifthispageodd}.
%    \begin{macrocode}
\newcommand*{\scr@tpo}{0}
%    \end{macrocode}
% \end{macro}^^A \scr@tpo
%
% \begin{macro}{\is@thispageodd}
% \changes{v2.8q}{2001/11/22}{added}
% \changes{v2.9l}{2003/01/28}{renamed \cs{if@thispageodd}}
% \changes{v2.95b}{2006/07/28}{using \cs{@newl@bel}}
% \changes{v3.17}{2015/03/23}{split into two macros}
% \begin{macro}{\is@thispageodd@setlabel}
% \changes{v3.17}{2015/03/23}{added}
% \changes{v3.20}{2016/03/25}{write to \file{aux} file only \cs{if@filesw}}
% \changes{v3.23}{2017/02/23}{empty macro after usage}
% The main workers of \cs{Ifthispageodd}. It is used to set
% \cs{ifthispagewasodd} depending on the information it adds to the \file{aux}
% file.
% \begin{description}
% \item[ToDo:] Why do we use an optional argument here? Are there any
%   circumstances, when the optional argument is not the default
%   \cs{is@thispageodd@setlabel}?
% \end{description}
%    \begin{macrocode}
\newcommand*{\is@thispageodd}[1][\is@thispageodd@setlabel]{%
  \@bsphack
  \begingroup
    \@tempcnta=\scr@tpo
    \advance\@tempcnta by\@ne
    \xdef\scr@tpo{\the\@tempcnta}%
    \expandafter\ifx\csname tpo@\scr@tpo\endcsname\relax
      \protect\G@refundefinedtrue
%<!extend>      \ClassWarning{\KOMAClassName}{%
%<extend>      \PackageWarning{scrextend}{%
        odd/even page label number \scr@tpo\space undefined}%
      \edef\@tempa{\the\value{page}}%
    \else
      \edef\@tempa{\csname tpo@\scr@tpo\endcsname}%
    \fi
    \ifodd\number\@tempa
      \aftergroup\thispagewasoddtrue
    \else
      \aftergroup\thispagewasoddfalse
    \fi
  \endgroup
  \edef\is@thispageodd@setlabel{%
    \if@filesw
      \unexpanded{\protected@write\@auxout{\let\arabic\relax}}{%
        \noexpand\string
        \noexpand\new@tpo@label{\scr@tpo}{\noexpand\arabic{page}}}%
    \fi
    \noexpand\def\noexpand\is@thispageodd@setlabel
    {\unexpanded\expandafter{\is@thispageodd@setlabel}}%
  }%
  #1%
  \@esphack
}
\newcommand*{\is@thispageodd@setlabel}{}
%    \end{macrocode}
% \end{macro}^^A \is@thispageodd@setlabel
% \end{macro}^^A \is@thispageodd
%
% \begin{command}{\thispagewasoddfalse,\thispagewasoddtrue,\ifthispagewasodd}
% \changes{v2.9l}{2003/01/20}{added}
% Rolf Niepraschks asked for the possibility to check for the last decision,
% e.g., inside the \meta{then code} or \meta{else code}. So the definition of
% \cs{Ifthispageodd} has been changed not only to set such a switch, but also
% use it.
%    \begin{macrocode}
\newif\ifthispagewasodd
%    \end{macrocode}
% \end{command}
%
% With all these macros the user command is very easy.
% \changes{v3.31}{2020/04/27}{fixed class and package name in the warning}
% \changes{v3.32}{2020/08/06}{\cs{KOMAClassName} fixed}
%    \begin{macrocode}
\providecommand*{\ifthispageodd}{%
%<class>  \ClassWarning{\KOMAClassName}{%
%<extend>  \PackageWarning{scrextend}{%
    Usage of deprecated command
    `\string\ifthispageodd'.\MessageBreak
    The command has been renamed because of a\MessageBreak
    recommendation of The LaTeX Project Team.\MessageBreak
    Please replace `\string\ifthispageodd' by `\string\Ifthispageodd'%
  }%
  \Ifthispageodd
}
\newcommand{\Ifthispageodd}{%
  \is@thispageodd
  \ifthispagewasodd
    \expandafter\@firstoftwo
  \else
    \expandafter\@secondoftwo
  \fi
}
%</(class|extend)&body>
%    \end{macrocode}
% \end{command}^^A \ifthispageodd,\Ifthispageodd
%
%
% \section{Implementation of Notes to the Margin}
%
% \begin{description}
% \item[ToDo:] This has been moved newly to
%   \file{scrkernel-miscellaneous.dtx}, but maybe we can find another suitable
%   file?
% \end{description}
%
% \begin{command}{\marginline}
% \changes{v2.4l}{1997/02/06}{using the optional argument of \cs{marginpar}}
% \changes{v2.8q}{2001/11/27}{missing brace added}
% \changes{v3.36}{2022/02/22}{moved from \file{scrkernel-floats.dtx} to
%   \file{scrkernel-miscellaneaous.dtx}}
% \changes{v3.39}{2022/12/21}{braces fixed}
% A very often wanted margin note feature is, to print all such notes in the
% left margin ragged-left and all such notes in the right margin
% ragged-right. So this is a simple shortcut for \cs{marginpar} with using
% almost the same argument twice.
%    \begin{macrocode}
%<*(class|extend)&body>
\newcommand\marginline[1]{%
  \marginpar[{\raggedleft #1}]{\raggedright #1}%
}
%</(class|extend)&body>
%    \end{macrocode}
% \end{command}
%
%
% \section{Implementation of Subscripts}
%
% \begin{command}{\textsubscript}
% \changes{v2.8h}{2001/07/19}{added}
% \changes{v3.25}{2017/11/15}{define only if un-defined because of \LaTeX{}
%   kernel change}
% \begin{macro}{\@textsubscript}
% \changes{v2.8h}{2001/07/19}{added}
% \changes{v3.10}{2011/09/05}{\cs{providecommand} instead of \cs{newcommand}
%   because of \pkg{fixltx2e}}
% The \LaTeX{} kernel has \cs{textsuperscript}. But for several years
% \cs{textsubscript} was missing. For such old \LaTeX{} kernels, we define it
% similar to \cs{textsuperscript}.
%    \begin{macrocode}
%<*(class|extend)&body>
\scr@ifundefinedorrelax{textsubscript}{%
  \DeclareRobustCommand*\textsubscript[1]{%
    \@textsubscript{\selectfont#1}%
  }%
}{%
  \scr@ifundefinedorrelax{textsubscript }{%
    \DeclareRobustCommand*\textsubscript[1]{%
      \@textsubscript{\selectfont#1}%
    }%
  }{}%
}
\providecommand*{\@textsubscript}[1]{%
  {\m@th\ensuremath{_{\mbox{\fontsize\sf@size\z@#1}}}}%
}
%</(class|extend)&body>
%    \end{macrocode}
% \end{macro}
% \end{command}^^A \textsubscript
%
%
% \section{Implementation of an Invisible Strut}
% \label{sec:strut}
%
% \begin{description}
% \item[ToDo:] Maybe \file{scrkernel-basics.dtx} would be a good place for
%   this and maybe is should become a user command and be used at several
%   places, currently \cs{strut} is used.
% \end{description}
%
% \begin{macro}{\scr@endstrut}
% \changes{v3.25}{2017/11/15}{added}
% If we simply add \cs{strut} to the end of material which is assumed to be
% horizontal, white spaces at the end of such material could become relevant,
% even if the would not without the \cs{strut}. We can avoid this by adding
% \cs{unskip} before the \cs{strut}. But in this case, the last work before
% the \cs{strut} will not be hyphenated any more. If we are really in
% horizontal mode, we can use \cs{nobreak} before the \cs{strut} you once
% again allow hyphenation.
% \begin{description}
% \item[ToDo:] Should the \cs{unskip} also be moved behind \cs{ifhmode}?
% \end{description}
%    \begin{macrocode}
%<*(class|letter|extend)&body>
\providecommand*{\scr@endstrut}{\unskip\ifhmode\nobreak\strut\fi}
%</(class|letter|extend)&body>
%    \end{macrocode}
% \end{macro}
%
%
% \section{Warnings for Using Incompatible Packages}
% \label{sec:packagewarning}
%
% \begin{description}
% \item[ToDo:] This should be moved to a new file.
% \end{description}
%
% \changes{v3.12}{2013/07/08}{warnings about several incompatible packages
%   added}
%
% \begin{macro}{\scr@package@not@recommended}
% \changes{v3.12}{2013/07/08}{added}
% \begin{macro}{\class@shoulddisablepackagewarning@list}
% \changes{v3.24}{2017/06/25}{added}
% Loading of several incompatible packages should result in warnings using:
% \begin{quote}
%   \cs{scr@package@not@recommended}\marg{package}\marg{alternatives}
% \end{quote}
% Usually the warning contains information about \meta{alternatives}, but this
% argument can also be empty. Authors of classes can deactivate the warning,
% by adding a \meta{package} to the comma separated list
% \cs{class@shoulddisablepackagewarning@lis}. This has to be done, before
% loading the \KOMAScript{} class. However doing so is not recommended and the
% class author is responsible for the drawbacks.
%    \begin{macrocode}
%<*class&body>
\newcommand*{\scr@package@not@recommended}[2]{%
  \scr@ifundefinedorrelax{class@shoulddisablepackagewarning@list}{%
    \@firstofone
  }{%
    \edef\reserved@a{%
      \noexpand\in@{#1}{\class@shoulddisablepackagewarning@list}}%
    \reserved@a
    \ifin@\expandafter\@gobble\else\expandafter\@firstofone\fi
  }{%
    \BeforePackage{#1}{%
      \ClassWarning{\KOMAClassName}{%
        Usage of package `#1' together\MessageBreak
        with a KOMA-Script class is not recommended.\MessageBreak
        \if\relax\detokenize{#2}\relax\else
          I'd suggest to use #2.\MessageBreak
        \fi
        Nevertheless, using requested\MessageBreak
        package `#1'%
      }%
    }%
  }%
}%
%    \end{macrocode}
% \end{macro}^^A \class@shoulddisablepackagewarning@list
% Here are the not recommended packages and the alternatives:
% \begin{description}
% \item[\pkg{emptypage}:] is not needed, because there are corresponding
% options and features in \KOMAScript.
%    \begin{macrocode}
\scr@package@not@recommended{emptypage}{%
  option\MessageBreak
  `cleardoublepage' with one of it's\MessageBreak
  several values}
%    \end{macrocode}
% \item[\pkg{fancyhdr}:] does not respect font settings of \KOMAScript{} and
%   also breaks some other features.
% \changes{v3.24}{2017/06/25}{improved \pkg{fancyhdr} warning}
%    \begin{macrocode}
\scr@package@not@recommended{fancyhdr}{%
  \MessageBreak
  package `scrlayer' or `scrlayer-scrpage', because\MessageBreak
  they support KOMA-Script classes.\MessageBreak
  With `fancyhdr' several features of class `\KOMAClassName'\MessageBreak
  like options `headsepline', `footsepline' or command\MessageBreak
  `\string\MakeMarkcase' and the commands `\string\setkomafont' and\MessageBreak
  `\string\addtokomafont' for the page style elements need\MessageBreak
  explicite user intervention to work}
%    \end{macrocode}
% \item[\pkg{minitoc}:]
% \changes{v3.23}{2017/02/04}{\pkg{minitoc} is not recommended}
% re-defines, e.g., \cs{@sect} and therefore breaks all the extensions of
% \KOMAScript{} to this command. Currently not all impacts of the package has
% been explored. Currently there is only a workaround for the \cs{@sect}
% issue. See \autoref{sec:packageworkarounds} for more information.
%    \begin{macrocode}
%<*!letter>
\scr@package@not@recommended{minitoc}{}
%</!letter>
%    \end{macrocode}
% \item[\pkg{multitoc}:]
% \changes{v3.26}{2018/07/18}{\pkg{multitoc} not recommended}
% At least up to version 2.01 this breaks all additional lists/tables of
% contents, because the \cs{@starttoc} redefinition does not allow to use
% another extension but \file{toc}, \file{lof}, or \file{lot}. The package is
% also not needed, because the \env{multicols} environment can simply be added
% using \cs{BeforeStartingTOC} and \cs{AfterStartingTOC}.
%    \begin{macrocode}
%<*!letter>
\scr@package@not@recommended{multitoc}{%
  package `multicol' and, e.g.:\MessageBreak
  \space\space
  \string\BeforeStartingTOC{\string\begin{multicols}{2}}\MessageBreak
  \space\space
  \string\AfterStartingTOC{\string\end{multicols}}\MessageBreak
  to make all ToCs two-columned.\MessageBreak
  \@ifpackagelater{2018/07/18}{}{%
    Note, package `multitoc' breaks additional\MessageBreak
    ToCs generated by `\string\DeclareNewTOC' or\MessageBreak
    packages like `listings' or `newcaption' and\MessageBreak
    even the standard table of contents,\MessageBreak
    list of figures or list of tables,\MessageBreak
    if you omit the corresponding options `toc',\MessageBreak
    `lof' or `lot'.%
  }%
}
%</!letter>
%    \end{macrocode}
% \item[\pkg{parskip}:] is an emergency fallback package for classes, that
%   do not support paragraph skips themselves.
%    \begin{macrocode}
\scr@package@not@recommended{parskip}{%
  option\MessageBreak
  `parskip' with one of it's several values}
%    \end{macrocode}
% \item[\pkg{titleps}:]
% \changes{v3.16}{2015/01/21}{\pkg{titleps} not recommended}
% does not respect several font settings and other features of the
% \KOMAScript{} classes.
%    \begin{macrocode}
\scr@package@not@recommended{titleps}{%
  package `scrlayer-scrpage'}
%    \end{macrocode}
% \item[\pkg{titlesec}:]
% \changes{v3.36}{2022/03/01}{bring back \pkg{titlesec} warning}
%   breaks several \KOMAScript{} features and needs a lot
%   of workarounds to not result in errors. Note, these workarounds have been
%   removed from \KOMAScript{}, because they are very fragile and use internal
%   macros of \pkg{titlesec}.
%    \begin{macrocode}
%<*!letter>
\scr@package@not@recommended{titlesec}{%
  KOMA-Script features\MessageBreak
  like \string\setkomafont, \string\RedeclareSectionCommand,
  \string\sectionlinesformat\MessageBreak
  and several others to re-configure section\MessageBreak
  headings}
%</!letter>
%    \end{macrocode}
% \item[\pkg{sectsty}:]
% \changes{v3.27}{2019/07/24}{\pkg{sectsty} up to v2.0.2 is incompatible}
% uses up to version 2.0.2 internal commands of
% \KOMAScript{}, that has been removed already 2001 in \KOMAScript{} 2.8e. So
% it was never compatible with \KOMAScript~3.
%    \begin{macrocode}
%<*!letter>
\AfterPackage*{sectsty}{%
  \@ifpackagelater{sectsty}{2002/02/26}{%
    \ClassInfo{\KOMAClassName}{Unknown `sectsty' release.\MessageBreak
      Cross your fingers, that is compatible%
    }%
  }{%
    \ClassWarning{\KOMAClassName}{%
      Incompatible package `sectsty' loaded!\MessageBreak
      Package `sectsty' up to version 2.0.2 is incompatible\MessageBreak
      with KOMA-Script from version 2.8e released in 2001.\MessageBreak
      You should avoid that package and configure the\MessageBreak
      section titles of your document using the KOMA-Script\MessageBreak
      user interface, e.g., \string\chapterlinesformat,\MessageBreak
      \string\RedeclareSectionCommand\space etc. or you should ask\MessageBreak
      the `sectsty' maintainer for a more compatible release.\MessageBreak
      Ignoring this warning could result in several error\MessageBreak
      messages and formatting issues after loading package\MessageBreak
      `sectsty'%
    }%
  }%
}
%</!letter>
%    \end{macrocode}
% \item[\pkg{tocbibind}:]
% \changes{v3.25}{2017/11/21}{Usage of \pkg{tocbibind} not recommended}
% \pkg{tocbibind} does several things, that break \KOMAScript, i.e.:
% \begin{itemize}
% \item usage of \cs{MakeUppercase} within running heads,
% \item usage of hard codes \texttt{toc} for the ToC entries,
% \item hard redefinition of \env{thebibliography},
% \item hard redefinition of \env{theindex},
% \item hard redefinition of \cs{tableofcontents},
% \item hard redefinition of \cs{listoftables},
% \item hard redefinition of \cs{listoffigures},
% \item hard redefinition of \cs{@makechapterhead}
% \end{itemize}
%    \begin{macrocode}
%<*!letter>
\scr@package@not@recommended{tocbibind}{%
  options like `listof=totoc'\MessageBreak
  or `bibliography=totoc', or commands like\MessageBreak
  `\string\setuptoc{toc}{totoc}' instead of this package,\MessageBreak
  because it breaks several KOMA-Script features of\MessageBreak
  the list of figures, list of tables, bibliography,\MessageBreak
  index and the running head%
}
%</!letter>
%    \end{macrocode}
% \item[\pkg{tocloft}:] 
% \changes{v3.26}{2018/07/18}{Usage of \textsf{tocloft} not recommended}
% redefines \cs{tableofcontents}, \cs{listoffigures},
% \cs{listoftables}, \cs{l@part} \dots{} \cs{l@subparagraph} and breaks
% \KOMAScript, i.e.:
% \begin{itemize}
% \item commands \cs{BeforeTOCHead}, \cs{AfterTOCHead},
%   \cs{BeforeStartingTOC}, \cs{AfterStartingTOC},
% \item almost all TOC features,
% \item options \opt{listof=totoc}, \opt{listof=numbered},
% \item option \opt{listof=flat},
% \item option \opt{listof=entryprefix},
% \item all options of \cs{DeclareTOCEntryStyle},
% \end{itemize}
%    \begin{macrocode}
%<*!letter>
\scr@package@not@recommended{tocloft}{%
  options like `listof=entryprefix',\MessageBreak
  commands like `\string\listoflofentryname' or\MessageBreak
  `\string\listoflotentryname',
  and `\string\DeclareTOCStyleEntry' or\MessageBreak
  `\string\RedeclareSectionCommand'
  instead of this package,\MessageBreak
  because it breaks several KOMA-Script features of\MessageBreak
  the list of figures, list of tables and table of\MessageBreak
  contents, i.e., options like `listof=numbered',\MessageBreak
  `listof=flat or `toc=flat', commands like\MessageBreak
  `\string\BeforeTOCHead{...}' and
  `\string\AfterTOCHead{...}',\MessageBreak
  `\string\BeforeStartingTOC{...}' and
  \string\AfterStartingTOC{...}',\MessageBreak
  all features of  `\string\DeclareTOCStyleEntry',\MessageBreak
  and the ToC entry features of `\string\DeclareSecionCommand'\MessageBreak
  and `\string\RedeclareSectionCommand' of levels\MessageBreak
  `part',
%<article> `chapter',
  `section', `subsection',\MessageBreak
  `subsubsection', `paragraph' and `subparagraph'%
}
%</!letter>
%</class&body>
%    \end{macrocode}
% \end{description}
% \end{macro}^^A \scr@package@not@recommended
%
% \section{Workarounds for Incompatible Packages}
% \label{sec:packageworkarounds}
%
% \begin{description}
% \item[ToDo:] The code should be moved to the same file as the code of
%   \autoref{sec:packagewarning} or maybe even to \pkg{scrhack}.
% \end{description}
%
% For some packages can be made compatible, patching some internal macros of
% these packages:
% \begin{description}
% \item[\pkg{minitoc}:]
%   \changes{v3.23}{2017/02/04}{\pkg{minitoc} is incompatible}
%   has some serious issues. The first issue is the hard re-definition of
%   \cs{@sect} without respecting the changes the \KOMAScript{} classes
%   already made. Here I'll try a ugly workaround for this, but is can
%   break. So not using the package with a \KOMAScript{} class would be the
%   better choice.
%    \begin{macrocode}
%<*class&body>
%<*!letter>
\BeforePackage{minitoc}{%
  \let\scr@mt@saved@sect\@sect
}
\AfterPackage{minitoc}{%
  \def\@tempa#1#2#3#4#5#6[#7]#8{%
    \ifnum #2<1 \relax 
      \addtocontents{toc}{\protect\sectbegin}
    \fi 
    \stc@sect{#1}{#2}{#3}{#4}{#5}{#6}[{#7}]{#8}%
  }%
  \ifx\@sect\@tempa
%    \end{macrocode}
% Note: There are several bugs in this definition (e.g. missing
% \cs{expandafter} and spurious spaced). Nevertheless its the original one
% from \pkg{minitoc} 61 and 62 and so I have to use it.
%    \begin{macrocode}
    \def\@tempa#1#2#3#4#5#6[#7]#8{
      \expandafter
      \ifx\csname #1\endcsname\section\relax
        \addcontentsline{lof}{xsect}{#7}%
        \addcontentsline{lot}{xsect}{#7}%
      \fi
      \ifx\csname #1\endcsname\starsection\relax
        \addcontentsline{lof}{xsect}{#7}%
        \addcontentsline{lot}{xsect}{#7}%
      \fi
      \ifnum #2>\c@secnumdepth\relax
        \let\@svsec\@empty
      \else
        \refstepcounter{#1}%
        \edef\@svsec{\csname the#1\endcsname\hskip 1em}%
      \fi
      \@tempskipa #5\relax
      \ifdim \@tempskipa>\z@
        \begingroup #6\relax
          \@hangfrom{\hskip #3\relax\@svsec}%
          {\interlinepenalty \@M #8\par}%
        \endgroup
        \csname #1mark\endcsname{#7}%
        \addcontentsline{toc}{#1}{%
          \ifnum #2>\c@secnumdepth\relax
          \else
            \protect\numberline{\csname the#1\endcsname}%
          \fi
          #7%
        }%
      \else
        \def\@svsechd{#6\hskip #3\relax
          \@svsec #8\csname #1mark\endcsname{#7}%
          \addcontentsline{toc}{#1}{%
            \ifnum #2>\c@secnumdepth\relax
            \else
              \protect\numberline{\csname the#1\endcsname}
            \fi
            #7}%
        }%
      \fi
      \@xsect{#5}%
    }%
    \ifx\stc@sect\@tempa
      \ClassWarning{\KOMAClassName}{%
        Redefining minitoc's internal command
        `\string\stc@sect'\MessageBreak
        to prevent minitoc to destroy several features\MessageBreak
        of KOMA-Script%
      }%
      \let\stc@sect\scr@mt@saved@sect
    \else
      \ClassInfo{\KOMAClassName}{%
        Deactivating the `minitoc' workaround,\MessageBreak
        because of unexpected definition of\MessageBreak
        `\string\stc@sect'%
      }%
      \let\scr@mt@saved@sect\relax
    \fi
  \else
    \ifx\@sect\scr@mt@saved@sect
      \ClassInfo{\KOMAClassName}{%
        Deactivating the `minitoc' workaround for
        `\string\@sect',\MessageBreak
        because it is not needed%
      }%
    \else
      \ClassInfo{\KOMAClassName}{%
        Deactivating the `minitoc' workaround,\MessageBreak
        because of unexpected definition of\MessageBreak
        `\string\@sect'%
      }%
    \fi
    \let\scr@mt@saved@sect\relax
  \fi
}
%</!letter>
%</class&body>
%    \end{macrocode}
% \item[\pkg{titlesec}:]
%   \changes{v3.13a}{2014/10/09}{\pkg{titlesec} is hardly incompatible}
%   \changes{v3.27}{2019/07/23}{\pkg{titlesec} does not support
%     KOMA-Script classes}%
%   After several years \textsf{titlesec} still does not really support
%   \KOMAScript{} classes and \KOMAScript{} would still need an ugly hack
%   using internal \texttt{titlesec} macros and deactivating several features
%   of \KOMAScript{} to make the package work. So IMO the better solution is
%   to remove such hacks from the \KOMAScript{} classes. Maybe the
%   \pkg{titlesec} author will support the \KOMAScript{} classes in some
%   future, maybe not. If you really need to \pkg{titlesec} with a
%   \KOMAScript{} class, see the ugly \texttt{standardsection} hack in
%   \pkg*{scrhack}.
% \end{description}
%
%
% \Finale
% \PrintChanges
% 
\endinput
% Local Variables:
% mode: doctex
% ispell-local-dictionary: "en_US"
% eval: (flyspell-mode 1)
% TeX-master: t
% TeX-engine: luatex-dev
% eval: (setcar (or (cl-member "Index" (setq-local TeX-command-list (copy-alist TeX-command-list)) :key #'car :test #'string-equal) (setq-local TeX-command-list (cons nil TeX-command-list))) '("Index" "mkindex %s" TeX-run-index nil t :help "makeindex for dtx"))
% End:
